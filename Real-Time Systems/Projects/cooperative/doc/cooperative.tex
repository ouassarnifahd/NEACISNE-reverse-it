\documentclass[a4paper]{report}

\usepackage[utf8]{inputenc}
\usepackage[T1]{fontenc}
\usepackage[french]{babel}

\usepackage{graphicx}
\usepackage{wrapfig}
\usepackage{geometry}
\usepackage{appendix}
\usepackage{array}
\usepackage{verbatim}
\usepackage{caption}
\usepackage{listings}
\usepackage[table]{xcolor}
\usepackage[explicit]{titlesec}
\usepackage{fancyhdr}
\usepackage{lastpage}
\usepackage{geometry}
\usepackage{amsmath,amsfonts,amssymb}
\usepackage[sfdefault]{ClearSans}
\usepackage{caption}
\usepackage{ifthen}
\usepackage{tocloft}
\usepackage{appendix}
\usepackage{calc}
\definecolor{ensigray}{RGB}{123,148,158}
\definecolor{ensiblue}{RGB}{88,141,200}
\definecolor{ensidarkgray}{RGB}{68,84,106}
\geometry{hmargin=2.5cm,vmargin=3.5cm}

\setcounter{tocdepth}{3}
\setcounter{secnumdepth}{4}
\renewcommand{\thesection}{\arabic{section}.}
\renewcommand{\thesubsection}{\arabic{section}.\arabic{subsection}.}
\renewcommand{\thesubsubsection}{\arabic{section}.\arabic{subsection}.\arabic{subsubsection}.}

% \titleformat
% {\chapter}[display]{\normalfont\sffamily\Huge}{}{0em}{#1}
% [\color{ensigray}\vspace{-0.75em}\rule{\textwidth}{1pt}]

\titleformat
{\section}[display]{\normalfont\sffamily\Large\color{ensigray}}{}{0em}{\thesection~#1}[]
\titleformat
{name=\section,numberless}[display]{\normalfont\sffamily\Large\color{ensigray}}{}{0em}{#1}[]
\titleformat
{\subsection}[block]{\normalfont\sffamily\large\color{ensigray}}{}{0em}{\thesubsection~#1}[]
\titleformat
{name=\subsection,numberless}[block]{\normalfont\sffamily\large\color{ensigray}}{}{0em}{#1}[]
\titleformat
{name=\subsubsection}[block]{\normalfont\sffamily\normalsize\color{ensigray}}{}{0em}{\thesubsubsection~#1}[]
\titleformat
{name=\paragraph}[block]{\normalfont\sffamily\normalsize\color{ensigray}}{}{0em}{~#1}[]

\captionsetup{labelsep=period,labelfont={color=ensidarkgray,it,small},textfont={color=ensidarkgray,it,small}}

\pagestyle{fancy}
\fancypagestyle{plain}{%
\fancyhead{} % clear all header fields
\fancyfoot{} % clear all footer fields
\fancyfoot[RO]{%
  \begin{picture}(0,0)
    \color{ensigray}
    \linethickness{2.25pt}
    \put(0,14){\line(1,0){500}}
  \end{picture}%
\@title\hspace{12pt}/{\color{ensigray}\thepage}}
\fancyfoot[LE]{\hspace{0.25cm}ENSICAEN /{\color{ensigray}\thepage}%
\begin{picture}(0,0)
    \color{ensigray}
    \linethickness{2.25pt}
    \put(0,14){\line(-1,0){200}}
\end{picture}}
}

\pagestyle{plain}

\renewcommand{\headrulewidth}{0pt}
\renewcommand{\footrulewidth}{0pt}

\fancyhfoffset[L]{\oddsidemargin + 1.5cm}
\fancyhfoffset[R]{\evensidemargin + 1.5cm}

\renewcommand{\arraystretch}{1.5}
\renewcommand{\baselinestretch}{1.5}

%
% Aspect des titres des "Table des matières/figures/tableaux"  
%
\renewcommand{\cftaftertoctitle}{\par\color{ensigray}\vspace{-6pt}\rule{\textwidth}{2pt}\vspace{-1cm}}
\renewcommand{\cfttoctitlefont}{\Huge}
\renewcommand{\cftafterloftitle}{\par\color{ensigray}\vspace{-6pt}\rule{\textwidth}{2pt}\vspace{-1.5cm}}
\renewcommand{\cftloftitlefont}{\Huge}
\renewcommand{\cftafterlottitle}{\par\color{ensigray}\vspace{-6pt}\rule{\textwidth}{2pt}\vspace{-1.5cm}}
\renewcommand{\cftlottitlefont}{\Huge}
\setlength\cftbeforetoctitleskip{0cm}
\setlength\cftbeforeloftitleskip{1cm}
\setlength\cftbeforelottitleskip{1cm}

\newcommand{\MakeSureLastPageIsEven}{
\clearpage%
\ifthenelse{\isodd{\value{page}}}%
        {\thispagestyle{empty}~\newpage}%
        {}%
}

%%% Local Variables: 
%%% mode: latex
%%% TeX-master: "reportV0.2"
%%% End: 

\input{dep/dirview.tex}

\begin{document}

\title{FreeRTOS - Mode cooperatif}
\author{OUASSARNI Fahd}
% \maketitle

% couverture
\makeatletter
\newgeometry{top=1.8cm,bottom=2cm,height=25.9cm,width=19cm,left=1cm}
\begin{titlepage}
    \begin{center}
        \huge\color{ensigray}École Publique d'Ingénieurs en 3 ans
    \end{center}
    \vspace{1.2in}
    \begin{center}
        \LARGE Travaux Pratiques
    \end{center}
    \vspace{0.5cm}
    \begin{minipage}{1\linewidth}
        \centering
        {\Huge  \MakeUppercase{\@title}}
    \end{minipage}
    \vspace{1.2in}
    \begin{center}\large
        le 20 février 2018,

        version 1.1
    \end{center}
    \vspace{2em}

    \noindent\begin{minipage}[t]{0.47\linewidth}
      \begin{flushright}
        \large
        Fahd OUASSARNI\\
        {\color{ensiblue}ouassarni@ecole.ensicaen.fr}
      \end{flushright}
    \end{minipage}\hfill\vrule\hfill
    \begin{minipage}[t]{0.47\linewidth}
      \large
      Tuteur~: Hugo DESCOUBES\\
      {\color{ensiblue}email@ensicaen.fr}
    \end{minipage}
    \vfill

    % logo
    \begin{center}
        \begin{minipage}[c]{0.33\linewidth}
          \center\includegraphics[width=0.8\linewidth]{dep/LogoEnsicaen2017_crop.png}\\[1em]
          {\color{ensigray}\huge www.ensicaen.fr}
        \end{minipage}
    \end{center}
\end{titlepage}
\restoregeometry

\newgeometry{top=3.4cm,bottom=3.4cm,left=2.5cm,width=16cm}
\setlength{\footskip}{2cm}



%%%%%%%%%%%%%%%%%%%%%%%%%%%%%%%%%%%%%%%%%%%%%%%%%%%%%%%%%%%%%%%%%%%%%%%%%%%%%%%%

\section*{Introduction}
\addcontentsline{toc}{section}{Introduction}

Lorem ipsum dolor sit amet, consectetur adipiscing elit, sed do eiusmod tempor
incididunt ut labore et dolore magna aliqua. Ut enim ad minim veniam, quis
nostrud exercitation ullamco laboris nisi ut aliquip ex ea commodo consequat.

Lorem ipsum dolor sit amet, consectetur adipiscing elit, sed do eiusmod tempor
incididunt ut labore et dolore magna aliqua.

Lorem ipsum dolor sit amet, consectetur adipiscing elit, sed do eiusmod tempor
incididunt ut labore et dolore magna aliqua. Ut enim ad minim veniam, quis
nostrud exercitation ullamco laboris nisi ut aliquip ex ea commodo consequat.

%%%%%%%%%%%%%%%%%%%%%%%%%%%%%%%%%%%%%%%%%%%%%%%%%%%%%%%%%%%%%%%%%%%%%%%%%%%%%%%%

\renewcommand{\cftdot}{}
\newpage

% \titleformat
% {\chapter}[display]{\normalfont\sffamily\Huge}{}{0em}{#1}
% [\color{ensigray}\vspace{-0.75em}\rule{\textwidth}{2pt}]

\tableofcontents
% \listoffigures       % Ligne à supprimer en l'absence de figures
% \listoftables        % Ligne à supprimer en l'absence de tableaux
% \endgroup

\makeatletter
\def\@makeschapterhead#1{%
  {\parindent \z@ \raggedright
    \interlinepenalty\@M
    \normalfont\sffamily\Huge \MakeUppercase{#1}\par\vskip-14pt\nobreak
    \color{ensigray}\vspace{-0.75em}\rule{\textwidth}{2.25pt}
    \vskip 16\p@
  }}
\makeatother

%%%%%%%%%%%%%%%%%%%%%%%%%%%%%%%%%%%%%%%%%%%%%%%%%%%%%%%%%%%%%%%%%%%%%%%%%%%%%%%%

\chapter*{Travail préparatoire}
\addcontentsline{toc}{chapter}{Travail préparatoire}

% \begin{center}
%   \begin{minipage}{0.7\linewidth}
%     \color{ensigray}\center
%     Lorem ipsum dolor sit amet, consectetur adipiscing elit, sed do eiusmod tempor
%     incididunt ut labore et dolore magna aliqua. Ut enim ad minim veniam, quis
%     nostrud exercitation ullamco laboris nisi ut aliquip ex ea commodo consequat.
%   \end{minipage}
% \end{center}

\section{Quels états peu prendre une tâche sous FreeRTOS ?}

Lorem ipsum dolor sit amet, consectetur adipiscing elit, sed do eiusmod tempor
incididunt ut labore et dolore magna aliqua. Ut enim ad minim veniam, quis
nostrud exercitation ullamco laboris nisi ut aliquip ex ea commodo consequat.

\section{Ces états sont-ils les mêmes quelque soit l'OS ou le RTOS utilisé ?}

Lorem ipsum dolor sit amet, consectetur adipiscing elit, sed do eiusmod tempor
incididunt ut labore et dolore magna aliqua. Ut enim ad minim veniam, quis
nostrud exercitation ullamco laboris nisi ut aliquip ex ea commodo consequat.

\section{Que se passe-t-il sous FreeRTOS lorsqu'aucune tâche, précédemment créée via xTaskCreate(), ne s'exécute (état running) ?}

Lorem ipsum dolor sit amet, consectetur adipiscing elit, sed do eiusmod tempor
incididunt ut labore et dolore magna aliqua. Ut enim ad minim veniam, quis
nostrud exercitation ullamco laboris nisi ut aliquip ex ea commodo consequat.

\section{Qu'est-ce qu'un TCB ?}

Lorem ipsum dolor sit amet, consectetur adipiscing elit, sed do eiusmod tempor
incididunt ut labore et dolore magna aliqua. Ut enim ad minim veniam, quis
nostrud exercitation ullamco laboris nisi ut aliquip ex ea commodo consequat.

\section{Que trouve-t-on en général dans un TCB (prendre l'exemple de FreeRTOS) ?}

Lorem ipsum dolor sit amet, consectetur adipiscing elit, sed do eiusmod tempor
incididunt ut labore et dolore magna aliqua. Ut enim ad minim veniam, quis
nostrud exercitation ullamco laboris nisi ut aliquip ex ea commodo consequat.


%%%%%%%%%%%%%%%%%%%%%%%%%%%%%%%%%%%%%%%%%%%%%%%%%%%%%%%%%%%%%%%%%%%%%%%%%%%%%%%%
\chapter*{TP: Mode cooperatif}
\addcontentsline{toc}{chapter}{TP: Mode cooperatif}

% \section{Repertoire du TP}

\begin{forest}
  pic dir tree,
  pic root,
  for tree={% folder icons by default; override using file for file icons
    directory,
  },
  [cooperative
    [doc
      [cooperative.pdf, file
      ]
    ]
    [inc
      [utask.h, file
      ]
      [FreeRTOSConfig.h, file
      ]
    ]
    [lib
      [FreeRTOS
      ]
      [peripherals
        [uart
        ]
      ]
    ]
    [src
      [utask.c, file
      ]
      [ktrap.c, file
      ]
      [main.c, file
      ]
    ]
    [readme.txt, file
    ]
  ]
\end{forest}

\end{document}
